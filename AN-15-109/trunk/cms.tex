%%%%%%%%%%%%%%%%%%%%%%%%%%%%%%%%
\section{Detector and trigger}
\label{sec:CMS}

The central feature of the CMS apparatus is a superconducting solenoid of 
6\unit{m} internal diameter, providing a magnetic field of 3.8\unit{T}. 
Within the superconducting solenoid volume are a silicon pixel and strip tracker, 
a lead-tungstate crystal electromagnetic calorimeter, and a brass{\slash}scintillator 
hadron calorimeter, each composed of a barrel and two endcap sections. 
Muons are measured in gas-ionization detectors embedded in the steel flux-return 
yoke outside the solenoid. Extensive forward calorimetry complements the coverage 
provided by the barrel and endcap detectors. 
%-- old
%The central feature of the CMS apparatus is a superconducting solenoid,
%13\unit{m} in length and 6\unit{m} in diameter, 
%that provides an axial magnetic field of 3.8\unit{T}.
%The core of the solenoid is instrumented with various particle detection
%systems: a silicon pixel 
%and strip tracker, an electromagnetic calorimeter (ECAL), and a
%brass/scintillator hadron calorimeter 
%(HCAL).  
%
%The silicon pixel and strip tracker covers $|\eta|<2.5$, ECAL and HCAL cover 
%$|\eta|<3$, and a quartz-steel Cerenkov-radiation-based forward hadron calorimeter extends the
%coverage to $|\eta| \leq 5$. 
%
The detector is nearly hermetic, covering $0<\phi<2\pi$ in azimuth, and thus allows
the measurement of momentum balance in the plane transverse to the beam
direction.
%
The first level of the CMS trigger system, composed of custom hardware
processors, uses information from the calorimeters and muon detectors to
select the most interesting events in a fixed time interval of less than
4~$\mu$s. The high level trigger processor farm further decreases the event
rate, from around 100\unit{kHz} to around 300\unit{Hz}, before data storage.
%The first level of the CMS trigger system, composed of custom hardware %Marcela's
%processors, uses information from the calorimeters and muon detectors to
%select the most interesting events. The High Level Trigger (HLT) utilizes a
%processor farm to further decrease the event rate.
%
A more detailed description of the CMS detector, together with a definition of the 
coordinate system used and the relevant kinematic variables, 
can be found in Ref.~\cite{Chatrchyan:2008aa}. 


%%%%%%%%%%%%%%%%%%%%%%%%%%%%%%%%
